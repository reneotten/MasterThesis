\chapter{Introduction}
\label{chap:intro}

\section{Motivation}
Peaking with the recent release of the IBM quantum experience interest in quantum computing, or more general quantum information processing, has recently been on the rise. In addition to the scientific community the general public has become increasingly aware leading to programs like the EU quantum effort. Further industry leaders like IBM, Google and Microsoft have shown great interest sparking major collaborations with university all around the globe. 

Quantum information processing holds the potential to revolutionize modern computing by employing basic quantum mechanical principles namely superposition, tunneling and entanglement. This would allow for efficient simulation of quantum mechanical systems, leading to the field of quantum simulation, but also provide an exponential speedup with some more common task like prime factorization using Shors (quantum) algorithm. 
In this so called \textit{Second Quantum Revolution} artificial quantum mechanical systems are employed to solve the tasks mentioned above.

The incredible potential held by quantum information processing has led to a zoo of different approaches trying to create qubits, which are the fundamental building blocks making up a quantum computer. The most important ones that are still actively researched are ion trap qubits, superconducting qubits and electron spin qubits without there being a clear winner. 
On the theoretical side there  has been extensive work on the conditions for a working quantum computer leading to the so called DiVincenzo criteria:
\begin{enumerate}
    \item A scalable physical system with well characterised qubits.
    \item The ability to initialise the state of the qubits to a simple fiducial state.
    \item Long relevant decoherence times.
    \item A “universal” set of quantum gates.
    \item A qubit-specific measurement capability.
\end{enumerate}
To also allow for quantum communication two more criteria have to be added:
\begin{itemize}
    \item The ability to interconvert stationary and flying qubits.
    \item The ability to faithfully transmit flying qubits between specified locations.
\end{itemize}
(see \rfig{fig:layout}). 
At the time of this thesis high fidelity single and two qubit gates have been shown with many different approaches and material systems, while first multi qubit systems have been explored.
In this thesis I will focus on two-electron spin qubits in GaAs, one of the most mature semiconductor based systems, with coherence times exceeding $\si{200.\micro s}$



\section{Goal}
In this thesis, I will take some of the essential steps on the way towards an experimental realization of exchange driven two qubit gates in GaAs in order to perform Gate optimization based on tuning protocols previously employed in our group. 
High Fidelity two qubit gates are a necessary requirement for a quantum computer in any material system. With its high coherence times, all electrical control and straight forward fabrication GaAs is the perfect material system for the development of methods and algorithms that can be transferred to other material systems later.


\section{Outline}

 

